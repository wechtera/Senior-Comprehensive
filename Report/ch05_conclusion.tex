%
% $Id: conclusion.tex
%
%   *******************************************************************
%   * SEE THE MAIN FILE "AllegThesis.tex" FOR MORE INFORMATION.       *
%   *******************************************************************


\chapter{Discussion and Future Work}\label{ch:conclusion}

This section covers and discusses the possible field of future works, and broadly overviews and reviews some of the issues, discoveries, and information delivered by the other aspects of this paper.

\section{Summary of Results}
As mentioned before, the results of this project were relatively promising.  The system was able to optimize the traffic grid up to 35\% more than the starting fitness in some cases.  While these solutions could be more accurate with the use of a more accurate simulator, this is not a bad starting point and should encourage further research in the field.  It was also observed that several of the variables tested for performance, had little effect on the outcomes of the optimization process.  An exception was the mutation rate, where lower values resulted in greater optimization.  Finally, the only major variation in values stemmed from weighing the different variables for the fitness function differently.  Even with these variations, overall trends remained consistent.

\section{Future Work}
The most easily achievable improvement to this project would be a better representation of the map and traffic system.  This would include a more realistic map, while still constrained to the grid system, involving more accurate speed limits for these roads, car destination and starting point hot spots, a better direction choosing system, and a better method of having cars enter and exit the system.  The speed limits and car starting and destination choices would be very simple, yet time-intensive jobs.  It would entail limits within the city being set for urban, housing, and commercial sections,  better approximating the way cities exist in real life.  Concurrently, it would be beneficial for a better direction setting algorithm to be used in place of breadth-first search.  Djikstra's algorithm is a possible replacement, because it calculates weight, or the speed limit of a road in this situation, as a factor while deciding on the best path.  Finally, allowing cars to enter and exit the system at different times so that there wasn't a flood of them entering the system at once would be an improvement.  This could be achieved by keeping a global time within the system and incorporating an entrance time variable for each car.   Exiting the system in a more elegant manner would be slightly more involved and require additional variables and checking to be made for how far in a road a car was and needed to be to be considered at its destination.  Doing this in a generic way would be slightly more complex.

Besides the above simple steps for increasing the accuracy of this project's system, there are other areas which can be improved in future work.  These are primarily limited to having more accurate simulation models and using these models on more accurate real city plans.  As mentioned, the primary drawback of this study was the amount of time that could be diverted to the development of the simulator.  If the simulation were more accurate, then indications of how well the genetic algorithm is able to perform would also be much more accurate.  Enhancements towards realism would include the use of irregular sized intersections (more or less than 4 lanes), as well as factors pertaining to the characteristics and behaviors of the cars.  Additionally, incorporating aspects such as rush hour, weather, driver personalities, road construction, and other driving irregularities would render the simulator more realistic.  The simulator also, as discussed earlier, is in charge of evaluating the fitness of the configuration.  It would be interesting to see if a new fitness could be developed which was able to more accurately portray how well the traffic configuration was able to influence and improve the conditions within the system.  

A final area of future work for this project might be the comparison of this project's methods with several of the afore-mentioned algorithms in the related works section.  Although this method was able to successfully optimize systems, the degree to which they were improved could perhaps be increased through the use of other algorithms, or partial genetic algorithms which use some heuristic to avoid local minima and maxima.  Comparing the results of this algorithm, and the results of other algorithms used on the same system would provide a better idea on how well this heuristic was able to optimize the system.

\section{Conclusion}
As a review, this study had the goal of showing that genetic algorithms, when coupled with a simulator to evaluate fitness, are able to adequately optimize a traffic grid to minimize several resultants of the system.  These include time delay, total time transit, and gas usage of the cars within the system.  To do this, a simulator was built which would take a perfect square as a grid pattern, incorporate speed limits and other aspects, introduce a number of randomly or personally configured cars, and implement an intersection configuration file to run as a discrete event simulator.  The simulations eventually resulted in a fitness numbers which represented the different important values itemized earlier, that were then used to indicate how well the current light configuration performed.  This simulator was then paired with the ECJ evolutionary algorithm suite to implement a genetic algorithm which generated and eventually proposed a more optimal traffic configuration.

To test the performance of these components, an exhaustive suite of trials was run impacting the mutation rate, crossover rate, and modifier weights, related to the fitness function.  These were run through a bash script and the results compiled and analyzed.  The findings indicate that genetic algorithms and this projects approach may be a valid and powerful tool to optimizing a large traffic grid through manipulation of the intersection traffic controls only.  Additionally none of the trialed values had a major impact on the results of the trials apart from the mutation rate.  In this case, it was found that having a lower mutation rate of 0 was the most powerful.  This trend was produced across all other configurations of the genetic algorithm

This study encourages research to move further into the presented problem of optimizing traffic control configurations to minimize congestion and delays.   Implementing different genetic algorithms is suggested.  Perhaps using multi-objective genetic algorithm would yield better results.  Finally, making a comparative study between this implementation and other heuristic algorithms would allow the findings of this study to be compared with other approaches and thus result in more meaningful conclusions.  